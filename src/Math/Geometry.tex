%! Author = jacobvetere
%! Date = 7/15/22

% Preamble
\documentclass[12pt]{article}


\usepackage{amsmath}
\usepackage{outlines}
\usepackage[english]{babel}
\usepackage[utf8x]{inputenc}
\usepackage[T1]{fontenc}
\usepackage{listings}
\usepackage{tikz}
\usepackage{graphicx}

% Graphics
\graphicspath{{./graphics/}}

\newcommand{\defline}[2]{\noindent\textbf{\underline{#1}}: #2\\}
% Document
\begin{document}

    \section {Geometry Theorems}
        `The laws of nature are but the mathematical though of god` - Euclid
        \subsection{Terms and Labels}
        \begin{outline}[enumerate]
            \1 \defline{Point}{Ordered pair of numbers representing a point on a plane}
                \2 (x,y) where x is the value along the horizontal axis and y is the value on the vertical axis
                \2 0 Dimensions
            \1 \defline{Line segment}{Line connecting two ordered pairs}
                \2 If A and B are points on the same plane then $\overline{AB}$ is the direct path between Point A and Point B
                \2 1 Dimension
                \2 No width only length
                \2 Length can be found several ways
                    \3 if $A = (x_1, y_1)$ and $B = (x_2, y_2)$
                    \begin{equation}
                        \sqrt{(x_1^2 - x_2)^2 + (y_1^2 - y_2)} = L_{\overline{AB}}
                    \end{equation}
            \1 Ray: A directed line from an arbitrary point to another
                \2 Cannot extend further from the point defined as the Vertex but can continue in a linear path from T (Transit)
                \2 Notation: $\overrightarrow{AD}$ Where A is the vertex
                \2 Ray is a form of directed line segment
            \1 Line: Bi-directional Ray that can continue past endpoints
                \2 Notation: $\overleftrightarrow{AD}$
            \1 Collinear: If we have $\overline{xy}$ and we add a z in between $x$ and $y$ these lines are now collinear
                \2 if $|\overline{XZ}| = |\overline{XY}|$ and $X, Y and Z$ are collinear then the
                length of segment $\overleftrightarrow{XY}$ is equal to the length of $|\overline{XZ}| * 2$
                \2 This indicates that the segment has a mid-point at Z
            \1 Planes/Planer: A two-dimensional flat infinite space
                \2 Movement left or right location on planar is represented as $(x, y)$
            \1 Image of Transformation: The form of shape after a transformation
        \end{outline}
        \begin{outline}
            \1 Rigid Transformation: Transformation that do not act on the distance between two points in the graph
            \1 Transformation
                \2 Translate: Shift entrypoint like this
                \2 Where $P$ is set an ordered pair of coordinates $(x_n,\ y_n)$
                \2 For any transformation there can be two modifiers $n$ and $m$
                \2 The new points can then be represented as $(x_n\ +\ n,\ y_n\ +\ m)$
            \1 Rotation
                \2 Pick any arbitrary point either in the set $P$ or on the plane and rotate all points about it
            \1 Reflection
                \2 Reflect across a line
                \2 The line acts as the mid point of symmetry therefore the new Image will be a mirror for of the transformed shape

        \end{outline}
\end{document}