%! Author = jacobvetere
%! Date = 7/15/22

% Preamble
\documentclass[12pt]{article}


\usepackage{amsmath}
\usepackage{outlines}
\usepackage[english]{babel}
\usepackage[utf8x]{inputenc}
\usepackage[T1]{fontenc}
\usepackage{listings}
\usepackage{tikz}
\usepackage{graphicx}

% Graphics
\graphicspath{{./graphics/}}

\newcommand{\defline}[2]{\noindent\textbf{\underline{#1}}: #2\\}
% Document
\begin{document}

    \section {Geometry Theorems}
        `The laws of nature are but the mathematical though of god` - Euclid
        \subsection{Terms and Labels}

            \defline{Point}{Ordered pair of numbers representing a point on a plane}
            \begin{outline}
                \1 (x,y) where x is the value along the horizontal axis and y is the value on the vertical axis
                \1 0 Dimensions
            \end{outline}

            \defline{Line segment}{Line connecting two ordered pairs}
            \begin{outline}
                \1 If A and B are points on the same plane then $\overline{AB}$ is the direct path between Point A and Point B
                \1 1 Dimension
                \1 No width only length
                \1 Length can be found several ways
                    \2 if $A = (x_1, y_1)$ and $B = (x_2, y_2)$
                    \begin{equation}
                        \sqrt{(x_1^2 - x_2)^2 + (y_1^2 - y_2)} = L_{\overline{AB}}
                    \end{equation}
            \end{outline}

            \defline{Ray}{A directed line from an arbitrary point to another}
            \begin{outline}
                \1 Cannot extend further from the point defined as the Vertex but can continue in a linear path from T (Transit)
                \1 Notation: $\overrightarrow{AD}$ Where A is the vertex
                \1 Ray is a form of directed line segment
            \end{outline}

            \defline{Line}{Bi-directional Ray that can continue past endpoints}
            \defline{Notation}{$\overleftrightarrow{AD}$}

            \defline{Collinear}{If we have $\overline{xy}$ and we add a z in between $x$ and $y$ these lines are now collinear}
            \begin{outline}
                \1 if $|\overline{XZ}| = |\overline{XY}|$ and $X, Y and Z$ are collinear then the
                length of segment $\overleftrightarrow{XY}$ is equal to the length of $|\overline{XZ}| * 2$
                \1 This indicates that the segment has a mid-point at Z
            \end{outline}

            \defline{Planes/Planer}{A two-dimensional flat infinite space. Movement left or right location on planar is represented as $(x, y)$}

            \defline{Image of Transformation}{The form of shape after a transformation}

        \subsection{Transformation}

            \defline{Rigid Transformation}{Transformation that do not act on the distance between two points in the graph}

            \defline{Transformation}{The process of shifting entrypoint in such a way as this}
            \begin{outline}
                \1 Translate:
                \1 Where $P$ is set an ordered pair of coordinates $(x_n,\ y_n)$
                \1 For any transformation there can be two modifiers $n$ and $m$
                \1 The new points can then be represented as $(x_n\ +\ n,\ y_n\ +\ m)$
            \end{outline}

            \defline{Rotation}{Pick any arbitrary point either in the set $P$ or on the plane and rotate all points about it}

            \defline{Reflection} {Reflect across a line The line acts as the mid point of symmetry therefore the new Image will be a mirror for of the transformed shape}
    \section{}
\end{document}