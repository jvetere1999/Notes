%! Author = jacobvetere
%! Date = 7/25/22

% Preamble
\documentclass[12pt]{article}


\usepackage[english]{babel}
\usepackage{amsmath}
\usepackage{amssymb}
\usepackage{amsthm}
\usepackage{array}
\usepackage{outlines}
\usepackage{mathtools}

\newcolumntype{C}{>$c<$}

\newtheorem{theorem}{Theorem}[section]
\newtheorem{lemma}[theorem]{Lemma}
\renewcommand\qedsymbol{QED}

\usepackage{enumitem}
\setlist[enumerate,2]{label=\roman*)}
\setlist[enumerate,3]{label=\alph*)}

\newcommand{\defline}[2]{\noindent\textbf{\underline{#1}}: #2\\}
\newcommand{\theorem}[2]{\noindent\textbf{\underline{Theorem #1}}: #2\\}
\newcommand{\example}[2]{
    \begin{equation}
        #1
    \end{equation}\caption{#2}\\
}
\newcommand{\transitionExamples}[4]{
    \begin{align*}
        #1 && \Rightarrow #3\\
        \text{#2} && \text{#4}
    \end{align*}
}
\newcommand{\transition}[2]{
    \begin{align*}
        #1 && \Rightarrow #2\\
    \end{align*}
}
\newcommand{\transitionLine}[1]{
    \begin{align*}
        #1 && \Rightarrow
    \end{align*}
}
\newcommand{\arrEntry}[2]{
    \[
        #1 & #2\\
        \hline
    \]
}
\newcommand{\setBuilder}[3]{
    $\{\isIn{#1}{#2}\ |\ #3\}$
}
\newcommand{\isIn}[2]{
    $#1\in #2$
}
\newcommand{\contradiction}{$\rightarrow\leftarrow$}
\newcommand{\symbolicexample}[3]{
    \begin{theorem}
        #1
    \end{theorem}

    \\\\
    Initial Proposition
    \begin{equation}
        #2
    \end{equation}

    \\\\
    Negated proposition
    \begin{equation}
        #3
    \end{equation}
}
% Document
\begin{document}

    \section{Logic}

        \defline{Logic/Logical Reasoning}{A system for drawing conclusions from a premise}

        \defline{Premise}{Input for logical reasoning, output is conclusion}

        \defline{Conclusion}{Truth dependent outcome}

        \defline{Connectives}{Operators for logical statements}

        \defline{Propositions}{Often referred to as a sentence, it is a declarative expression which is either true or false}

        \defline{Logical Equivalence}{A logically equal expression will have a truth table with the same last row as the comparator}

        \begin{left}
            \[
                \begin{array}{|C|C|}
                    \hline
                    Def & Math\\
                    \hline
                    not & \neg\\
                    \hline
                    and & \wedge\\
                    \hline
                    or & \vee\\
                    \hline
                    if... then ... &  \rightarrow\\
                    \hline
                    iff/If and only if... & \leftrightarrow\\
                    \hline
                    XOR/Exclusive Or & \#\\
                    \hline
                    Universal Quantifier & \forall\\
                    \hline
                    Existential Quantifier & \exists\\
                    \hline
                    Improper Subset & \subseteq\\
                    \hline
                    Proper Subset & \subset\\
                    \hline
                    Contradiction & \rightarrow\leftarrow\\
                    \hline
                \end{array}
            \]
        \end{left} \caption{Connectives/Formal Logical Operators}

    \endsection

    \section{Sets}

        \defline{Tautology}{A sentence that is always True such as}

        \begin{equation}
            p\vee\neg{p}
        \end{equation}


        \defline{Contradiction}{A proposition that is always False}

        \begin{equation}
            p\wedge\neg{q}
        \end{equation}

        \defline{Sets}{Informally a collection of elements}

        \defline{Elements}{Objects which are contained in a set}
        \subsection{Set Notation}
            \begin{outline}
                \1 $\{1,2,3,4\}$
                \1 The set of positive even numbers
                \1 $\{[3,4],[1,2]\}$
            \end{outline}
        \subsection{Common Sets}
            \begin{left}
                \[
                    \begin{array}{|C|C|}
                        \hline
                        Symbol & Set\\
                        \hline
                        \mathbb{N} & Natural Set \{1,2,3,...\}\\
                        \hline
                        \mathbb{Z} & Set of all integers \{x : -\infty-\infty\}\\
                        \hline
                        \mathbb{Q} & Set of irrational numbers ie proper and improper functions\\
                        \hline
                        \mathbb{R} & Set of Real numbers, decimal\\
                        \hline
                        \emptyset & Empty Set\\
                        \hline
                    \end{array}
                \]
        \end{left}
    \endsection

    \section{Quantifiers}

        \defline{Quantifiers}{Used to make mathematical language more precise}

        \defline{Universal Quantifier}{$\forall$ means for all, for every or for each}

        \example{(\forall x \in \mathbb{N})(x+1\in \mathbb{N})}{For any value $x$ in the natural set the value $x+1$ is also in the natural set}

        \defline{Existential Quantifier}{$\exists$ means there exists, there is, for some, or there is at least one}

        \example{(\exists x\in\mathbb{N})(x > 5)}{For some value x in the natural set x is greater than 5}

        \defline{Order Of Quantifiers}{Very important for example}

        \example{(\forall x \in \mathbb{N})(\exists x\in \mathbb{N})(y>x)}{For each $x\in\mathbb{N}$, there exists $z\in\mathbb{N}$ such that $y>x$ - Statement 1}

        \example{(\exists y\in\mathbb{N})(\forall x\in\mathbb{N})(y>x)}{There exists $z\in\mathbb{N}$ such that for each $x\in\mathbb{N}$, $y>x$ - Statement 2}

        \defline{Statement 1}{States that given any positive integer there exists a larger integer this is true. Since $\forall$ is before $\exists$ the value $y$ is dependent on x}

        \defline{Statement 2}{States that there is a possible integer which is greater than all others this is false.}
    \endsection

    \section{Negating Quantified Statements}
        \defline{Negation}{Let A be a set, for all $x\in A$ let $P(x)$ be the statement}

        \example{\neg(\forall x \in A)(P(x))}{It is false that for all $x\in A$, $P(x)$ is true}

        \example{(\exists x \in A)(\neg P(x))}{There is at least one $x\in A$ for which $P(x)$ is false}

        \subsection{Method for Negating Quantifiers}

        \begin{outline}
            \1 At front of sentence
                \2 $\forall \Rightarrow \exists$
                \2 $\exists \Rightarrow \forall$
            \2 Negate proposition governed by quantification
        \end{outline}

        \transitionExamples{(\forall x\in \mathbb{N})(x\in \mathbb{Z})}{For each $x\in \mathbb{N}, x\in \mathbb{Z},$}{(\exists x\in \mathbb{N})(x\notin\mathbb{Z})}{There exists $x\in \mathbb{N}$ where $x\notin \mathbb{Z}$}
        
        \transition{(\forall x \in \mathbb{Z})(\exists y\in \mathbb{Z})(x+y=0)}{(\exists x\in \mathbb{Z})(\forall y\in \mathbb{Z})(x+y\ne 0)}

        \transition{(\forall x\in \mathbb{Q})(\forall y\in \mathbb{Q})(x>y) \rightarrow (\exists x\in \mathbb{Q})(x>z>y)}
        {\\\\(\exists x\in \mathbb{Q})(\exists x\in \mathbb{Q})((x>y)\wedge(\forall z\in \mathbb{Q}((x\leq z)\vee(z\leq y))))}
        Explanation Pg 16

        \begin{left}
            \[
                \begin{array}{|C|C|}
                    \hline
                    Start & End\\
                    \hline
                    \neg (p \rightarrow q) & p\wedge \neg q\\
                    \hline
                    \neg (p > q) & p \leq q\\
                    \hline
                    \forall & \exists\\
                    \hline
                    \exists & \forall\\
                    \hline
                    \in & \notin\\
                    \hline
                \end{array}
            \]
        \end{left}
    \endsection
    \section{Subsets and Set Builder Notation}
        \subsection{Subsets}
            \defline{Subsets}{Let $A$ and $B$ be sets. $A$ is a subset of $B$ if (and only if) for all $x\in A$, $x\in B$.
            This is stated as $A$ is a subset of $B$}

            \example{(\forall x \in A)(x\in A)\leftrightarrow(x\in B)}{}

            \defline{Note}{If can be used inplace of if and only if}

            \defline{Not a Subset}{$A\not\subseteq B$. There exists some $x\in A$ for which $x\notin B$ then $A\not\subseteq B$ or rather}
            \example{(\exists x\in A)(x\notin B)\leftrightarrow(A\not\subseteq B)}{Not a subset of B}

            \defline{Set Builder Notation}{Let $A$ be a set, for all $x\in A$, let $P(x)$ be a proposition about $x$ we can write this set as}
            \example{S = \{x\in A | P(x)\}}{The set of all $x\in A$ such that $P(x)$}

            \example{B = \{1, 2,...,5\} \Rightarrow B = \{x\in \mathbb{N}\ |\ x\leq 5\}}{}

            \example{E =\{x\in N\ |\ (\exists k\inZ)(n=2k)\}}{Let $E$ be the set of all even positive integers}

            \example{\emptyset = \{x\in\mathbb{N}\ |\ (x > 2)^(x < 2)\}}{Empty Set}

            \defline{Universe of Discourse}{Any set used in a discussion are subsets of some set $\mathbb{X}$. ie Calc $\rightarrow R$}

            Let $\mathbb{X}$ be a Universe of Discourse and let $(A\subseteq \mathbb{X}) \wedge (B\subseteq \mathbb{X})$\\

            \defline{Intersection}{Denoted as $A\cap B$, it is the set $\{x\in \mathbb{X}\ |\ x\in A \wedge x\in B\}$}

            \defline{Union}{$A \cup B$ is denoted as $\{x\in \mathbb{X}\ |\ x\in A \vee x\in B\}$}

            \defline{Relative Complement}{The relative complement of B in A, denoted as $A - B$ is the set $\{x\in \mathbb{X}\ |\ x\in A \wedge x\notin B\}$}

            \defline{Complement of a Universe of Discourse}
            {Let $\mathbb{X}$ be a known Universe of Discourse and let $A\subseteq \mathbb{X}$. The relative complement of $A$ in $\mathbb{X}$ s just called the complement and denoted as $A^c$}

            \subsection{Set Op Examples}
                Let $X = \{1, 2, 3, 4, 5, 7, 8\}$, let $A = \{1, 3, 5\}$ and let $B = \{5, 9\}$ then\\

                \\$A \cap B = \{5\}$

                \\$A \cup B = \{1, 3, 5, 9\}$

                \\$A - B = \{1,3\}$

                \\$B - A = \{9\}$

                \\$A^c =\{7, 9\}$

                \\$B^c = \{1, 3,7\}$\\


            \defline{Set Equality}{Set $A$ and $B$ are equal if and only if $A\subseteq B$ and $B \subseteq A$. In other words, $(A\subseteq B \wedge B\subseteq A) \iff (A = B)$}

    \newpage

    \section{Proof By Contradiction}
        \begin{theorem}
            For every set $A$, $\emptyset\subseteq A$
        \end{theorem}
        \begin{proof}
            By way of contradiction.
            Suppose there exists a set $A$ such that $\emptyset\not\subseteq A$, since $\emptyset\not\subseteq A$ there exists $x\in\emptyset$ such that $x\notin A$
            However the set $\emptyset$ contains no elements hence $x\notin \emptyset$ and $x\in \emptyset$.
            Therefore \contradiction.
            Our hypothesis has led to a contradiction and is therefore false.
            Thus for every set $A$, $\emptyset\subseteq A$.
        \end{proof}
        \begin{outline}
            \1 $(\exists A)(\emptyset\not\subseteq A)\Rightarrow$
            \1 $(\emptyset\not\subseteq A)\rightarrow(\exists x\in \emptyset)(x\notin A)\Rightarrow$
            \1 $(\forall x\in \mathbb{X})(x\notin \emptyset)$
            \1 $(x\in \emptyset)(x\notin\emptyset)$ is a \contradiction
        \end{outline}
        \subsection{Process of Proof By Contradiction}
            \defline{Proof by Contradiction}{Negate the proposition and find a contraction.
            Can be denoted as $(\neg P \rightarrow C)\rightarrow P$}

            \defline{$(\neg P \rightarrow C)\rightarrow P$}{A tautology can be used as a formal equation of sorts for an indirect proof.
            It states that: We assume $\neg P$ and show that $\neg P\rightarrow C$ where $C$ is a formal contradiction that proves the preposition.}

            \defline{QED Quod Erat Demonstrandum}{Latin for proof is finished}

            \begin{outline}[enumerate]
                \1 Statement of Process: By way of contradiction
                \1 Negation of Preposition: Negate logical preposition that is to be proved
                    \2 $(\forall A)(\emptyset\not\subseteq A) \xRightarrow{\neg}$
                        \3 Original preposition
                    \2 $(\exists A)(\emptyset\not\subseteq A)$
                        \3 Negation
                \1 Create logic chain to contradiction utilizing a negation of P if C implies P: $(\neg P \rightarrow C)\rightarrow P$
                \1 Restate Initial proposition
            \end{outline}
            \begin{theorem}
                For all sets $A$ and $B$, if $A\subseteq A\cap B$ then $A\cup B \subseteq B$
            \end{theorem}

            Initial Proposition
            \begin{equation}
                (\forall A)(\forall B)(A\subseteq A\cap B)\rightarrow(A\cup B \subseteq B)
            \end{equation}

            Negated Proposition
            \begin{equation}
                (\exists A)(\exists B)(A\subseteq A\cap B)\wedge (A\cup B \not\subseteq B)
            \end{equation}
            \begin{proof}
                By way of contradiction, there exists some set A and B for which $A\subseteq A\cap B$ and $A\cup B \not\subseteq B$.
                Since $A\cup B \not\subseteq B$ this implies that there exists $x\in A\cup B$ such that $x\notin B$.
                Since $x\in A\cup B$, $x\in A$ or $x\in B$.
                \\\\
                \defline{Case 1}{}
                Suppose that $x\in A$.
                Since $x\in A$ and $A\subseteq A\cap B$, $x\in A\cap B$.
                Since $x\in A\cap B$, $x\in A$ and $x\in B$.
                Thus $x\in B$. but by the choice of $x$, $x\notin B$. \contradiction`
                \\\\
                \defline{Case 2}{}
                Suppose $x\in B$.
                By the choice of $x$, $x\notin B$.
                Thus $x\in B$ and $x\notin B$ is a contraction \contradiction
                \\\\
                In both cases, our hypothesis has led to a contraction and is therefore false.
                Thus for all sets $A$ and $B$, if $A\subseteq A\cap B$ then $A\cup B\subseteq B$.
            \end{proof}

            \begin{outline}
                \1 $A \cup B\not\subseteq B\rightarrow (\exists x\in A\cup B )(x\notin B)$
                \1 $(x\in A\cup B)\rightarrow (x\in A)\vee (x\in B)$
                \1 Case 1:
                    \2 $(x\in A)\wedge (A\subseteq A\cap B) \rightarrow (x\in A\cap B)$
                    \2 $(x\in A\cap B)\rightarrow(x\in A)\wedge (x\in B)$
                    \2 Choice $x\notin B$ \contradiction $(x\in A)\wedge (x\in B)$
                \1 Case 2:
                    \2 $x\in B$ \contradiction Choice of $x\notin B$
                \1 $(\neg P \rightarrow C) \rightarrow P$
            \end{outline}

            \subsection{Examples}
                For each theorem, write the opening lines of a proof by contradiction.
                My challenge: Work and record the logical derivation to the contradiction and write the entire proof.
                \subsubsection{Problem 1}
                \symbolicexample{For all sets $A$ and $B$, if $A\subseteq B$ then $A-B=\emptyset$}
                {(\forall A)(\forall B)(A\subseteq B)\rightarrow (A-B=\emptyset)}
                {(\exists A)(\exists B)(A\subseteq B)\wedge (A-B\ne\emptyset)}

                Logical Derivation
                \begin{outline}
                    \1 $(\exists A)(\exists B)(A\subseteq B)\wedge (A-B\ne\emptyset)$
                        \2 $(A\subseteq B) \Rightarrow (x\in A)\leftrightarrow(x\in B)$
                        \2 $(A-B\ne\emptyset)\Rightarrow (x\in A)\wedge (x\notin B)$
                    \1 $((x\in A)\leftrightarrow(x\in B)) \wedge ((x\in A)\wedge (x\notin B))$`
                    \1 $x\in A \rightarrow (x\in B)$ \contradiction $(x\notin B)$
                \end{outline}
                \\\\
                \begin{proof}
                    By way of contradiction, there exists sets $A$ and $B$ such that $A\subseteq B$ and $A-B\ne \emptyset$.
                    Since $A\subseteq B$ implies that $x\in A$ if and only if $x\in B$.
                    Since $A-B\ne \emptyset$ this implies $x\in A$ and $x\notin B$.
                    \\\\
                    Now suppose $x\in A$ such that $x\in B$ and $x\notin B$.
                    This is a contradiction \contradiction.
                    Therefore, For all sets $A$ and $B$, if $A\subseteq B$, then $A-B=\emptyset$
                \end{proof}
                \\\\\\

                \subsubsection{Problem 2}
                \symbolicexample{For all sets $A$, $B$ and $C$, if $A\subseteq C$ and $B\subseteq C$ then $A\cup B \not\subseteq C$}
                {(\forall A)(\forall B)(\forall C)\ ((A\subseteq C)\wedge (B\subseteq C))\rightarrow (A\cup B \subseteq C)}
                {(\exists A)(\exists B)(\exists C)\ (A\subseteq C)\wedge (B\subseteq C)\wedge (A\cup B \not\subseteq C)}


    Logical Derivation
                \begin{outline}
                    \1 $(\exists A)(\exists B)(\exists C)\ (A\subseteq C)\wedge (B\subseteq C)\wedge (A\cup B \not\subseteq C)$
                        \2 $A\subseteq C\Rightarrow (x\in A)\leftrightarrow(x\in C)$
                        \2 $B\subseteq C\Rightarrow (x\in B)\leftrightarrow(x\in C)$
                        \2 $A\cup B \not\subseteq C\Rightarrow (x\in A\cup B)\leftrightarrow(x\notin C)$
                            \3 $\Rightarrow (x\in A)\wedge (x\in B)\leftrightarrow(x\notin C)$
                    \1 $(x\in A)\leftrightarrow(x\in C) \wedge (x\in B)\leftrightarrow(x\in C) \wedge (x\in A)\wedge (x\in B)\leftrightarrow(x\notin C)$
                    \1 $x\in A ^ x\in B \Rightarrow x\in C$ \contradiction $x\notin C$
                \end{outline}
                \\\\
                \begin{proof}
                    By way of contradiction, there exists sets $A$, $B$ and $C$ such that $A\subseteq C$ and $B\subseteq C$ and $A\cup B \not\subseteq C$.
                    Since $A\subseteq C$ then $x\in A$ if and only if $x\in C$, as well $B\subseteq C$ means that $x\in B$ if and only if $x\in C$.
                    However, since $A\cup B\not\subseteq C$, $x\in A \wedge x\in B$ if and only if $x\notin C$ which causes a contradiction. \contradiction
                    \\\\
                    Therefore, by contradiction, for all sets $A, B \text{ and }C$ if $A\subseteq C$ and $B\subseteq C$ then $A\cup B\subseteq C$.
                \end{proof}
    \endsection
\end{document}