%! Author = jacobvetere
%! Date = 7/25/22

% Preamble
\documentclass[11pt]{article}

% Packages
\usepackage{amsmath}
\usepackage{array}
\usepackage{outlines}
\usepackage{amsfonts}

\newcolumntype{C}{>$c<$}
\newcommand{\defline}[2]{\noindent\textbf{\underline{#1}}: #2\\}
\newcommand{\example}[2]{
    \begin{equation}
        #1
    \end{equation}\caption{#2}\\
}
\newcommand{\transitionExamples}[4]{
    \begin{align*}
        #1 && \Rightarrow #3\\
        \text{#2} && \text{#4}
    \end{align*}
}
\newcommand{\transition}[2]{
    \begin{align*}
        #1 && \Rightarrow #2\\
    \end{align*}
}
\newcommand{\transitionLine}[1]{
    \begin{align*}
        #1 && \Rightarrow
    \end{align*}
}
\newcommand{\arrEntry}[2]{
    \[
        #1 & #2\\
        \hline
    \]
}
% Document
\begin{document}

    \section{Logic}

        \defline{Logic/Logical Reasoning}{A system for drawing conclusions from a premise}

        \defline{Premise}{Input for logical reasoning, output is conclusion}

        \defline{Conclusion}{Truth dependent outcome}

        \defline{Connectives}{Operators for logical statements}

        \defline{Propositions}{Often referred to as a sentence, it is a declarative expression which is either true or false}

        \defline{Logical Equivalence}{A logically equal expression will have a truth table with the same last row as the comparator}

        \begin{left}
            \[
                \begin{array}{|C|C|}
                    \hline
                    Def & Math\\
                    \hline
                    not & \neg\\
                    \hline
                    and & \wedge\\
                    \hline
                    or & \vee\\
                    \hline
                    if... then ... &  \rightarrow\\
                    \hline
                    iff/If and only if... & \leftrightarrow\\
                    \hline
                    XOR/Exclusive Or & \#\\
                    \hline
                    Universal Quantifier & \forall\\
                    \hline
                    Existential Quantifier & \exists\\
                    \hline
                \end{array}
            \]
        \end{left} \caption{Connectives/Formal Logical Operators}

    \endsection

    \section{Sets}

        \defline{Tautology}{A sentence that is always True such as}

        \begin{equation}
            p\vee\neg{p}
        \end{equation}


        \defline{Contradiction}{A proposition that is always False}

        \begin{equation}
            p\wedge\neg{q}
        \end{equation}

        \defline{Sets}{Informally a collection of elements}

        \defline{Elements}{Objects which are contained in a set}
        \subsection{Set Notation}
            \begin{outline}
                \1 $\{1,2,3,4\}$
                \1 The set of positive even numbers
                \1 $\{[3,4],[1,2]\}$
            \end{outline}
        \subsection{Common Sets}
            \begin{left}
                \[
                    \begin{array}{|C|C|}
                        \hline
                        Symbol & Set\\
                        \hline
                        \mathbb{N} & Natural Set \{1,2,3,...\}\\
                        \hline
                        \mathbb{Z} & Set of all integers \{x : -\infty-\infty\}\\
                        \hline
                        \mathbb{Q} & Set of irrational numbers ie proper and improper functions\\
                        \hline
                        \mathbb{R} & Set of Real numbers, decimal\\
                        \hline
                        \emptyset & Empty Set\\
                        \hline
                    \end{array}
                \]
        \end{left}
    \endsection

    \section{Quantifiers}

        \defline{Quantifiers}{Used to make mathematical language more precise}

        \defline{Universal Quantifier}{$\forall$ means for all, for every or for each}

        \example{(\forall x \in \mathbb{N})(x+1\in \mathbb{N})}{For any value $x$ in the natural set the value $x+1$ is also in the natural set}

        \defline{Existential Quantifier}{$\exists$ means there exists, there is, for some, or there is at least one}

        \example{(\exists x\in\mathbb{N})(x > 5)}{For some value x in the natural set x is greater than 5}

        \defline{Order Of Quantifiers}{Very important for example}

        \example{(\forall x \in \mathbb{N})(\exists x\in \mathbb{N})(y>x)}{For each $x\in\mathbb{N}$, there exists $z\in\mathbb{N}$ such that $y>x$ - Statement 1}

        \example{(\exists y\in\mathbb{N})(\forall x\in\mathbb{N})(y>x)}{There exists $z\in\mathbb{N}$ such that for each $x\in\mathbb{N}$, $y>x$ - Statement 2}

        \defline{Statement 1}{States that given any positive integer there exists a larger integer this is true. Since $\forall$ is before $\exists$ the value $y$ is dependent on x}

        \defline{Statement 2}{States that there is a possible integer which is greater than all others this is false.}
    \endsection

    \section{Negating Quantified Statements}
        \defline{Negation}{Let A be a set, for all $x\in A$ let $P(x)$ be the statement}

        \example{\neg(\forall x \in A)(P(x))}{It is false that for all $x\in A$, $P(x)$ is true}

        \example{(\exists x \in A)(\neg P(x))}{There is at least one $x\in A$ for which $P(x)$ is false}

        \subsection{Method for Negating Quantifiers}

        \begin{outline}
            \1 At front of sentence
                \2 $\forall \Rightarrow \exists$
                \2 $\exists \Rightarrow \forall$
            \2 Negate proposition governed by quantification
        \end{outline}

        \transitionExamples{(\forall x\in \mathbb{N})(x\in \mathbb{Z})}{For each $x\in \mathbb{N}, x\in \mathbb{Z},$}{(\exists x\in \mathbb{N})(x\notin\mathbb{Z})}{There exists $x\in \mathbb{N}$ where $x\notin \mathbb{Z}$}
        
        \transition{(\forall x \in \mathbb{Z})(\exists y\in \mathbb{Z})(x+y=0)}{(\exists x\in \mathbb{Z})(\forall y\in \mathbb{Z})(x+y\ne 0)}

        \transition{(\forall x\in \mathbb{Q})(\forall y\in \mathbb{Q})(x>y) \rightarrow (\exists x\in \mathbb{Q})(x>z>y)}
        {\\\\(\exists x\in \mathbb{Q})(\exists x\in \mathbb{Q})((x>y)\wedge(\forall z\in \mathbb{Q}((x\leq z)\vee(z\leq y))))}
        Explanation Pg 16

        \begin{left}
            \[
                \begin{array}{|C|C|}
                    \hline
                    Start & End\\
                    \hline
                    \neg (p \rightarrow q) & p\wedge \neg q\\
                    \hline
                    \neg (p > q) & p \leq q\\
                    \hline
                    \forall & \exists\\
                    \hline
                    \exists & \forall\\
                    \hline
                    \in & \notin\\
                    \hline
                \end{array}
            \]
        \end{left}
    \endsection

\end{document}