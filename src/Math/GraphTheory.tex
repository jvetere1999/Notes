%! Author = jacobvetere
%! Date = 7/19/22

% Preamble
\documentclass[11pt]{article}

% Packages
\usepackage{amsmath}
\usepackage{outlines}
\usepackage[english]{babel}
\usepackage[utf8x]{inputenc}
\usepackage[T1]{fontenc}
\usepackage{listings}
\usepackage{tikz}
% Document
\begin{document}
    \maketitle{Graph Theorems}
    \begin{outline}
        \1 Three Friends/Three Strangers
            \2 Among any six people, there must be three mutual friends or three mutual strangers
            \2 Proof page 5
        \1 The First Theorem of Graph Theory (Handshake Lemma)
            \2 Let $G$ be a graph of order $n$ and size $m$ with vertices $v_1,\ v_2,\dots\ v_n$
            \2 Then $deg_{v_1}\ +\ deg_{v_1}\ +\ \dots\ + deg_{v_n}) = 2m$
            \2 a complete graph of order n has a size that can be calculated with $m\ =\ n(n-1)/2$
                \3 the number of given edges for a vertex $n$ in a complete graph $G$ is the number of vertices -1
                \3 Multiplying that by $n$ gives us the total number for all edges though counting every edge twice divide this by 2
        \1 Corollary: Every graph has an even number of odd vertices
            \2 Proof page 21
    \end{outline}
\end{document}